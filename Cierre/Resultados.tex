

A través del desarrollo del proyecto y el prototipo como tal se propuso un modelo funcional que simplifica y describe el proceso de contratación de un musico de manera clara y trazable. Lo anterior diferenciando los puntos de vista de los roles implicados en el proceso. El contratante que tiene la posibilidad de revisar ofertas, comunicarse con los servicios de su interés, y acordar un contrato. Y desde el punto de vista el Musico quien tiene la oportunidad de ofrecer sus servicios, recibir notificaciones de comunicación, aceptar contratos y conseguir retroalimentación de la prestación de su servicio.\\


El desarrollo produjo una herramienta que resuelve con facilidad la necesidad evidenciada; verificable por medio de escenarios de transacción que demuestran la funcionalidad del producto. Gracias a las técnicas de modelamiento de la arquitectura de software, adicionalmente se aseguran las cualidades de sostenibilidad; como los son mantenibilidad escalabilidad y la documentación del componente tecnológico producido.
