El propósito de generación del prototipo fue la generación de una herramienta de calidad que permita a los músicos gestionar sus servicios de una manera adecuada, sacando el mejor provecho de su actividad y delegando las responsabilidades que no son propias de su que hacer. Desde la perspectiva de la comunidad se planteo la idea de ofrecer la centralización de un servicio que hasta el para esta época no tiene normalización y que por lo mismo se encuentra disperso y difuso al momento de utilizarlo.\\

 A través de la navegación del prototipo demo se observa las concepción de centralización, logrando la oferta de servicios musicales, su promoción y divulgación, la comunicación entre interesados (oferente y cliente) y la concreción del servicio como tal.\\
 
Por medio del uso, aceptación y divulgación que obtenga la aplicación en el ambiente productivo se tendrá la oportunidad de cumplir con otro de los objetivos, que es directamente la promoción de talentos que se pueden desperdiciar por falta de apoyo.\\

En cuanto a tecnología se plantean escenarios ideales, teniendo en cuenta la etapa de diseño y fundamentos de arquitectura de software que producirán como resultado una herramienta solida escalable y sostenible con el tiempo. \\

El desarrollo del proyecto cumplió con las expectativas iniciales logrando la entrega del producto en el tiempo estimado, con las características apropiadas y descritas en las etapas iniciales, lo cual plantea una estructura de trabajo aplicada y con resultados verificables.
