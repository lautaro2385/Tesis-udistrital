\subsection{Marco Teórico}
\textbf{Antecedentes}\\
\textbf{Tu toque}, página web donde los músicos pueden ofrecer sus servicios con precios, formatos, géneros y estilos definidos; y los clientes pueden buscar, filtrar, comparar y contratar al músico ideal.
Esta plataforma permite, con base en los criterios de búsqueda del usuario, buscar por artista, ubicación, género musical y además filtrar por precios y calificaciones. Los perfiles de músicos ofrecen diferentes servicios y formatos musicales, lo cual brinda al cliente un gran número de opciones para elegir. Asimismo, el cliente cuenta con retroalimentación de clientes anteriores para formarse una mejor idea del servicio ofrecido por el músico.
Con el fin de dar tranquilidad a músicos y clientes, el pago se hace por anticipado, pero el músico recibe el dinero después de haber prestado el servicio, de manera que tuToque.co sirve como garante entre las partes \citeW{tutoque_2018} \citeW{tiempo_2018}. \\

\textbf{JamsOut}, Sitio web que funciona en Ecuador, es un buscador de artistas, permite a cantantes e instrumentistas colgar su información, más enlaces de sus trabajos en el portal.
La información debe constar de fotos y videos de sus conciertos, links de sus canciones en Spotify, iTunes o Dezeer. También debe colocar una reseña artística de 20 líneas, en la cual se hable de su trayectoria, escenarios donde ha cantado y una descripción del género musical en el que se desenvuelve.\\
 
\textbf{Inkasha}, es una red social en línea que busca promover el talento y la interacción entre bandas de Colombia a través de la creación de tres diferentes perfiles donde ellos pueden mostrar su música, descubrir trabajo, conectarse con otros músicos, y encontrar venues donde tocar.
Estos perfiles se dividen en músicos (solistas que busquen promover su música o que estén buscando una banda a la cual pertenecer), bandas ( que busquen participar en convocatorias, mostrar su música o encontrar integrantes para sus agrupaciones) y dueños de venues (auditorios, bares, cafés, hoteles, universidades, etc. donde busquen a músicos para presentarse en sus espacios).
Y los usuarios interesados pueden ingresar y mirar la información del artista y elegir si quieren sus servicios\citeW{rivera_2018}.\\


\textbf{Quiero músicos}, es una web colombiana donde se puede ofrecer los servicios de músicos, y los usuarios pueden buscar y requerir cotizaciones, se agrupan por tipo de eventos y se puede encontrar cualquier tipo de músico. \citeW{quieromusicos:online}\\
