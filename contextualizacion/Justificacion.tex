\section{Justificación del proyecto}
Se necesitan herramientas que puedan unir y comunicar todos los actores del problema, que son los artistas, el público, las entidades públicas y privadas, en la medida que los artistas necesitan del público, los escenarios necesitan eventos y artistas, y los sectores público y privado necesitan garantizar a la comunidad el acceso y la difusión a la cultura a través de una actitud de compromiso social que conlleve al objetivo del mejoramiento de la calidad de vida de las personas\citeW{bernal_2018}.\\

A los músicos se les recomienda tener un manager para establecer las relaciones comerciales, por que su formación no es de administrador, pero se han abierto un mundo a través de las alianzas que en muchos casos los proyectan en una labor más administrativa o artística. En casos donde los artistas estén en sus primeras fases, no va a ser necesario que ellos adquieran los servicios de un manager si a través de una plataforma digital enfocada hacia ellos, puedan promocionar sus servicios, mostrar sus talentos y tienen la posibilidad de ser contratados directamente\citeW{notas_2018}.\\

En el aspecto laboral, la informalidad y el desempleo en este sector son altos, y la oferta laboral presentada por cuenta de las industrias culturales es baja, y hay pocas organizaciones  que demanden los servicios de los artistas y en muchos casos no hay suficiente información sobre los artistas, sobre la calidad y los servicios que prestan; y la contratación muchas veces se hace por recomendaciones personales o contrataciones previas, lo que lleva a pérdidas de oportunidades de muchos artistas que por no ser conocidos o referenciados no llegan a ser contratados\citeW{murcia_2018}.\\

Identificada la necesidad de promoción de artistas, para evitar los intermediarios y que tengas más oportunidades laborales, este proyecto se orienta al  desarrollo de un prototipo de una aplicación móvil, en donde se van a servir de puente en los usuarios y los artistas.
