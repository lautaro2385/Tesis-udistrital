\section{Aspectos metodológicos}
\subsection{Tipo de estudio}
El tipo de investigación utilizado en la generación de este prototipo será proyectiva/interactiva\cite{holistica}. lo anterior teniendo en cuenta que la idea central del proyecto es proponer una nueva forma de realizar un proceso de contratación que modifique la que en este momento se utiliza de manera común, mejorandola y proyectándola a posiblemente impulsar nuevos talentos y apoyarlos en procesos administrativos y de negocios a los que no están acostumbrados comportándose como un manager para artistas informales que de pronto no cuenten con los recursos necesarios para adquirir dichos servicios.
\subsection{Método de investigación}
Para lograr el fin deseado utilizaremos apartes de cada uno de los niveles de la investigación holística de la siguiente manera:

\textbf{Perceptual}: Exploraremos las opciones utilizadas en la actualidad para lograr la contratación de un servicio de entretenimiento; describiéndolo de manera que encontremos sus falencias y oportunidades de mejora.

\textbf{Aprehensivo}: Compararemos las formas en las que se realiza el proceso de selección de artistas informales y analizaremos la que produzca mayor impacto para extraer de hay los puntos fuertes que deben incluirse en nuestra propuesta.

\textbf{Comprensivo}: Propondremos una nueva opción que cubra las necesidades de veladas en los pasos anteriores.

\textbf{Integrativo}: Apuntaremos a modificar la forma en que se realiza el proceso de selección al momento de elegir un artista informal para una presentación  

\subsection{Fuentes y técnicas para la recolección de la información}
La información primaria se obtendrá de la aplicación de una encuesta que determine la forma de contratación de servicios artísticos en una población controlada mayor de edad que esté en la capacidad de realizar dicho proceso, se generan preguntas cerradas que limitan las posibilidades de existentes y describan el proceso actual y la opción de cambiarlo.\\

Como segunda fuente se verificarán las bases de datos científicas especializadas en ingeniería para aplicar las metodologías necesarias en cuanto a la producción de software de calidad y que den valor agregado al producto final enmarcadas en procesos conocidos y de trazabilidad comprobable. Además de las generalidades legales y sociales que pueda acarrear el uso de material multimedia con propiedad intelectual. 

\subsection{Tratamiento de la información}

La información recopilada a través de la aplicación de la encuesta será anónima sin causar perjuicio a los originadores de la misma. Luego de ser obtenida se analizará para seleccionar las coincidencias importantes y generar las directrices de modelamiento del producto de software que se planea construir. 
Por otro lado la documentación técnica guía para el desarrollo del software será claramente referenciada en la bibliografía del producto dando el valor a las mismas y respetando los principios de propiedad intelectual.
