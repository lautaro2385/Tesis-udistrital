\section{Planteamiento del problema}

La situación económica del artista colombiano, en general, no ha sido nunca buena. Por supuesto, hay excepciones. Algunos pintores, escultores, músicos y escritores ganan mucho dinero. Son un puñado. En la pirámide hacia abajo un conglomerado sobrevive. Pero en la base una muchedumbre pasa las duras y las maduras. Si habláramos en términos de desempleo, la tasa artística podría alcanzar el 80 por ciento. Obviamente el desempleo entre los creadores de arte es relativo. Un creador podría ejecutar excelencias en un sótano oscuro en circunstancias de supervivencia dramáticas. De cualquier forma, la tendencia del estatus del artista en estos tiempos se inclina hacia la comodidad, el progreso técnico y el éxito económico. Por tanto, en este país, donde tras 23 años de lucha por obtener seguridad social inútilmente para el artista, no sorprende que dos grandes figuras se lancen a la aventura de la empresa artística, solo para abrirle posibilidades a la tasa de desempleados \citeW{murcia_2018}.\\

Muchos de estos artistas se han hecho a pulso, son muy talentosos pero no han podido concretar sus estudios universitarios en este arte. Comenzando por el costo de la carrera, que a diferencia del sector público (al que no todos pueden ingresar), en el privado, el precio por semestre oscila entre \$3 millones y \$11 millones; sumado a eso, los gastos de fotocopias, transporte, e instrumentos, que para cumplir medianamente con los estándares mínimos de calidad representan una cantidad de dinero considerable \citeW{benitez_2018}.\\

La informalidad laboral es un concepto que va más allá de la creencia de que solo las personas que trabajan en la calle o independientes son consideradas informales. Este término también abarca a los trabajadores a los que no se les ha legalizado su labor o a los que de alguna manera se les ha trasgredido alguno de los requisitos establecidos por la Organización Internacional del Trabajo (OIT) \citeW{Trabajoi81:online}.\\

Los músicos del rebusque intentan mantenerse —y mantener a sus familias— con lo que diariamente producen en las calles: pueden recoger entre 20 mil y 150 mil pesos en un solo día, todo depende del flujo de peatones y de la fecha, porque si pagaron la quincena la gente es más generosa. Muchos de estos intérpretes no pagan ni salud ni pensión.Según una encuesta realizada por el DANE, publicada el pasado 8 de abril, entre noviembre de 2015 y febrero de este año el empleo informal en Bogotá fue del 47,1\%. La cifra bajó tan solo un 1\% con respecto al mismo trimestre del año pasado \citeW{CARTELUR81:online}.\\

``basados en datos del Banco Interamericano de Desarrollo (BID), que la industria cultural aporta al Producto Interno Bruto (PIB) de Colombia un 3\%, pero señala que la inversión que tiene el Estado para la misma es de un 0,16\%, por lo que le parece paradójico que en Colombia no haya oportunidades suficientes para que la gente estudie o trabaje en ese campo": David García ex-director general de la Orquesta Filarmónica de Bogotá. \citeW{Derechoa87:online}.\\

``Estamos comprometidos con el impulso a la economía naranja para que nuestros actores, artistas, productores, \textbf{músicos}, diseñadores, publicistas, joyeros, dramaturgos, fotógrafos y animadores digitales conquisten mercados, mejoren sus ingresos, emprendan con éxito", dijo el presidente de Colombia Iván Duque, cuando tomó posesión del cargo \citeW{Colombia60:online}. Aún así, entre los retos que registra este modelo, según indica Duque, está en el de ``retener, atraer, capturar y reproducir el talento de un segmento de la población, que por lo general se encuentra subvalorado socialmente y pobremente remunerado económicamente” \citeW{Queesla74:online}, y acá es donde se quiere atacar el problema. \\

A través de la innovación facilitar la forma en que se ofrecen servicios ha sido una de las mayores utilidades que se le ha dado a internet, el problema que queremos solventar es especializar la oferta de un servicio que hasta el momento no ha sido visualizado. Aplicar todo el conocimiento de ingeniería de software para producir un producto con código limpio, escalable, probable y todas las características que lo identifiquen como de alta calidad. Planeamos realizar todo el ciclo de vida de un desarrollo aplicando metodologías y tecnologías de última generación que nos conduzcan a afianzar los conocimientos del arte tecnológico implicado en el proceso de producción de software.\\

La idea de generar un prototipo de aplicación móvil para centralizar la oferta y demanda de servicios musicales supone el facilitar el ambiente en el que se genera dicha interacción. Por lo anterior proponemos agregar a la aplicación un módulo de comunicación que permita el contacto directo entre oferente y demandante además de la posibilidad de historial de servicios con calificaciones que certifiquen la calidad del servicio prestado igualmente la calificación de los demandantes como personas confiables y respetables.


\section{Formulación del problema}
¿Cómo brindarles a los músicos  un prototipo de una herramienta digital implementada con las mejores prácticas de desarrollo de software  que facilite la búsqueda, dar a conocer y contratación de músicos informales o en fase de descubrimiento?

\section{Sistematización del problema}

\begin{itemize}
  \item ¿Cómo centralizar la oferta y demanda de servicios musicales para facilitar el ambiente en el que se genera dicha interacción?
  \item ¿Cuál es la arquitectura que más se ajusta para solucionar este tipo de problema?
  \item ¿Cómo mejorará los ingresos de los músicos asociados a la plataforma?
  \item ¿Cuál es la información más relevante que un usuario tiene en cuenta para contratar a un músico?
\end{itemize}





