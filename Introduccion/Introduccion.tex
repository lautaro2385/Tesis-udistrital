\setcounter{page}{1}
\chapter*{INTRODUCCIÓN}
\addcontentsline{toc}{chapter}{INTRODUCCIÓN}  
La realización de este proyecto supone un fondo social en el que se busca facilitar y mejorar la calidad de vida de las personas dedicadas a la prestación de servicios de entretenimiento que no tienen un conocimiento administrativo o de mercadeo adecuado para promocionar esta labor. Lo anterior toma mayor valor teniendo en cuenta la amplia oferta de programas educativos en programas universitario de música y arte que están produciendo gran número de egresados, que no encontrarán una demanda laboral favorable y se perderán en un mercado hostil que probablemente los llevará a desempeñarse en otras labores contrarias a su vocación. Por otra parte la informalidad que existe en el proceso de contratación de esta labor artística ha generado todo tipo de formas de contacto entre el oferente y el demandante y es por esto que el mercado es tan difuso e irregular. La centralización de la oferta permite mejorar la calidad y normaliza el mercado logrando así el mejor valor para los implicados en el proceso\citeW{benitez_2018}\citeW{bernal_2018}.\\ \\
 Para suplir dicha necesidad nos aventuramos en el proceso de desarrollo de una aplicación móvil innovadora enmarcada en los principios de desarrollo ágil aplicando los lineamientos de buenas prácticas. Lo anterior para fundamentar los conocimientos adquiridos a través de la experiencia en ingeniería de software.\\ \\
En este proceso ejecutamos el proceso investigativo que permite inferir la mejor opción para cubrir esta necesidad; realizando el análisis del proceso actual de selección de un servicio y contratación del mismo. Se realizará el modelamiento y diseño necesario inicial para la generación de un producto sólido y estable desde sus primeras versiones limitando la posibilidad de falla a casos imprevistos que se susciten solo a través del uso de la herramienta y no a la falta de planeación de las mismas.
